%-------------------------
% Resume in Latex
% Author : Saradha Venkatachalapathy
% License : MIT
%------------------------

\documentclass[letterpaper,11pt]{article}

\usepackage{latexsym}
\usepackage[empty]{fullpage}
\usepackage{titlesec}
\usepackage{marvosym}
\usepackage[usenames,dvipsnames]{color}
\usepackage{verbatim}
\usepackage{enumitem}
\usepackage[pdftex]{hyperref}
\usepackage{fancyhdr}
\usepackage{hyperref}
\usepackage{ragged2e}
\usepackage{etaremune}
\pagestyle{fancy}
\fancyhf{} % clear all header and footer fields
\fancyfoot{}
\renewcommand{\headrulewidth}{0pt}
\renewcommand{\footrulewidth}{0pt}

% Adjust margins
\addtolength{\oddsidemargin}{-0.375in}
\addtolength{\evensidemargin}{-0.375in}
\addtolength{\textwidth}{1in}
\addtolength{\topmargin}{-.5in}
\addtolength{\textheight}{1.0in}

\urlstyle{same}

\setlist{nosep}
\setlist[itemize]{leftmargin=0.7cm}
\setlength{\tabcolsep}{0in}

\hypersetup{
  colorlinks   = true, %Colours links instead of ugly boxes
  urlcolor     = [rgb]{0.3,0.3,0.5}, %Colour for external hyperlinks
  linkcolor    = [rgb]{0.3,0.3,0.5} %Colour of internal links
  citecolor   =  [rgb]{0.3,0.3,0.5} %Colour of citations
}
% Sections formatting
\titleformat{\section}{
  \vspace{-4pt}\scshape\raggedright\large
}{}{0em}{}[\color{black}\titlerule \vspace{-5pt}]

%-------------------------
%%%%%%  CV STARTS HERE  %%%%%%%%%%%%%%%%%%%%%%%%%%%%

\begin{document}

%----------HEADING-----------------
\begin{flushleft}
{\LARGE {\bf Saradha Venkatachalapathy}} \hfill  {\href{mailto:saradhavpathy@gmail.com}{saradhavpathy@gmail.com}}\\
{ }\hfill {\href{https://saradhavenkatachalapathy.github.io/}{\underline {Website}}}{ }
  {\textbar}{ }{\href{https://www.linkedin.com/in/saradhav/}{\underline {LinkedIn}}} \\
 \end{flushleft}
%-----------SUMMARY-----------------
\section{\bf SUMMARY}
\justify{ Ph.D. student at the Mechanobiology Institute, NUS specializing in genomics and computer vision. Extensive background in developing algorithms and implementing statistical models to interpret causality in highly variable processes. Experienced in leading interdisciplinary collaborative projects with biologists and mathematicians to guide the design of experiments and modeling.
}


%-----------EDUCATION AND WORK EXPERIENCE-----------------
\vspace{0.8mm}
\section{\bf EDUCATION AND WORK EXPERIENCE}
{\bf Ph.D, Mechanobiology }{\textbar}{ National University of Singapore }{\textbar}{ GPA 4.7/5} \hfill {\em Sep 2016 - Present} 
\\
{\bf B.Tech Biotechnology }{(Distinction) }{\textbar}{ SASTRA University }{\textbar}{ GPA 8.1/10} \hfill {\em Jul 2011 - May 2015} 
\\
{\bf Consultant}{, Computer Vision }{\textbar}{ }\href{https://www.qritive.com/}{\underline {Qritive}} \hfill {\em Sep 2019 - Dec 2019} 
\\
{\bf Research Assistant}{, National University of Singapore }
\hfill {\em Sep 2015 - Jul 2016} 

%--------SKILLS------------
\vspace{0.8mm}
\section{\bf SKILLS}
{\bf Statistics: }{Machine Learning, Linear Algebra, Regression, Diffusion maps and Pattern recognition.}
\\
{\bf Computer Vision: }{Segmentation, Feature generation and Particle tracking}
\\
{\bf Computational Biology: }{Analysis of bulk and single cell Microarray, RNA-Seq and HiC.}
\\
{\bf Experimental Skills: }{Microscopy, Tissue engineering and mechanical manipulation of cells.}
\\
{\bf Tools: }{R, MATLAB, Python (pandas, scikit, PyTorch), SQL, Git, LATEX and Inkscape.}
%-----------SELECTED RESEARCH PROJECTS-----------------
\vspace{0.8mm}
\section{\bf SELECTED RESEARCH PROJECTS}
{\bf Automated feature generator for 3D images}
\begin{itemize}
    \item Built an in-house automatic image processing pipeline for segmentation and feature generation.
    \item Developed novel parameters that measure morphology, textural and spatial distribution of objects.
     \item Integrated multi-domain features such as protein expression, RNA seq and image features for deducing functional links. 
    \item This reduced the processing time from 48 hours to 3 hours. 
\end{itemize}
\vspace{1mm}
{\bf Deconvolving cell variability in cancer}
\begin{itemize}
    \item Only a subset of cells are activated by cancer signals in engineered breast cancer tissue.
     \item Developed a linear classifier to predict cell shape with an accuracy of 95\%.
     \item Established the existence of activation primed cell shapes using multimodal-multivariate analysis.
    \item Demonstrated a causal relationship between cell geometry and activation \em {[MBoC,2020]}.
\end{itemize}
\vspace{1mm}
{\bf Time series analysis of reprogramming}
\begin{itemize}
    \item Developed a novel method to reprogram fibroblasts to iPSC-like cells.
    \item Aligned, analyzed and visualized the transcriptome during mechanically induced de-differentiation. Performed statistical tests and pathway analysis to characterize the temporal changes in the transcription profile and infer the biological relevance \em {[PNAS,2018]}.
    \item Implemented pseudo-temporal ordering of single cell data to identify variable trajectories during the generation of stem cells. 
\end{itemize}

%--------HONORS AND AWARDS------------
\vspace{1mm}
\section{\bf HONORS AND AWARDS}
\begin{itemize}
    \item Dean’s Merit list given to the top 2-10\% students in the University \hfill \emph {2015}
    \item Inspirational Mentorship Award, NUS High School \hfill \emph {2017} 
    \item Best Oral Presentation Award, Genomes and AI: From Packing to Regulation \hfill \emph{2019} 
\end{itemize}

\vspace{2mm}

Complete List of publications {\href{https://tinyurl.com/pulications-sv}{\underline {Here}}}

\end{document}







